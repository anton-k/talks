% csoundexpression-Ac.tex
\begin{hcarentry}[updated]{csound-expression}
\report{Anton Kholomiov}%05/15
\status{active, experimental}
\makeheader

The csound-expression is a Haskell framework for electronic music production.
It's based on very efficient and feature rich synth Csound.
It strives to be as simple and responsive as it can be.
Features include almost all Csound build in audio units support,
composable GUIs, FRP for event scheduling, MIDI and OSC support and many others.
The library was updated for GHC-7.10.

200+ beautiful instruments are implemented. See the csound-catalog package.
Each instrument is ready for real-time usage. Three drum machines are implemented.
There is a library of standard effects. It can be used as a guitar processor.

With Csound it inherits many cutting edge sound synth techniques like
granular synthesis or hyper vectorial synthesis, ambisonics.

The csound-expression is a Csound code generator. The flexible nature of Csound 
(it's written in C and has wonderful API)  allows to use the produced 
code on any desktop OS, Android, iOS, Raspberry Pi, Unity, within many other languages.
We can create audio engines with Haskell.

The library was presented at the Russian Function programming conference 2015
and at the International Csound Conference 2015. 

The future plans for the library is to bring it on stage and make 
some audio installations with it, to improve documentation.
I've created some music with the library. You can listen to it 
on the soundcloud \url{https://soundcloud.com/anton-kho}.

The library is available on Hackage.
See the packages csound-expression, csound-sampler and csound-catalog.

\FurtherReading
  \url{https://github.com/anton-k/csound-expression}
\end{hcarentry}
